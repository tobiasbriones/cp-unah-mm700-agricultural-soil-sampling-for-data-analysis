\documentclass[conference]{IEEEtran}
\usepackage[utf8]{inputenc}
\usepackage[spanish]{babel}
\usepackage[letterpaper, portrait, margin=2cm]{geometry}
\usepackage[style=ieee]{biblatex}
\usepackage{amsthm}
\usepackage{amsmath}
\usepackage{amssymb}
\usepackage{amsfonts}
\usepackage{hyperref}
\usepackage{csquotes}
\usepackage{mathtools}
\usepackage{graphicx}
\usepackage{float}
\usepackage{array}
\usepackage{dirtytalk}
\usepackage[table,xcdraw]{xcolor}

\addbibresource{bibliography.bib}

\newtheorem{definition}{Definición}

\DeclarePairedDelimiter{\ceil}{\lceil}{\rceil}

\title{
MODELO DE MUESTREO VIRTUAL DE SUELOS AGRÍCOLAS EN HONDURAS PARA ANALÍTICA EN BASE AL MUESTREO ALEATORIO SIMPLE Y ESTRATIFICADO
}

\author{
\IEEEauthorblockN{Tobias Briones}
\IEEEauthorblockA{\textit{Universidad Nacional Autónoma de Honduras} \\
\textit{Carrera de Licenciatura en Matemática}\\
San Pedro Sula, Honduras \\
tobias.briones@unah.hn} \\\vspace*{20pt} \normalsize  
Diciembre 2021
}
\date{Septiembre 2021}

% ----- THIS DOCUMENT CONTAINS ALL THE BLUEPRINT PIECES TOGETHER -----

\begin{document}

\maketitle

\begin{center}
    \includegraphics[width=0.3\linewidth]{ref/logo-unah.png}\\[4ex]
    \textit{Universidad Nacional Autónoma de Honduras}
    
    \bigbreak
    
    \textit{Proyecto de Seminario de Investigación Presentado a la Carrera de Licenciatura en Matemática}
\end{center}

\thispagestyle{empty}
\pagestyle{empty}
\tableofcontents
\listoffigures

% ------------------------------  FRAMEWORK  ------------------------------ %



% ------------------------------ END OF  FRAMEWORK ------------------------------ %

\section{Anexos}

Se ha tomado nota en pequeña parte sobre la elaboración y estructura de \textit{Manual de Elaboración y Presentación de Tesis} \cite{universidad-san-carlos-2016}, \textit{Metodología de la investigación} \cite{collado-2014}, \textit{Introducción a la metodología de la investigación científica} \cite{cabezas-2018} para esta propuesta de investigación.

\subsection{Glosario}

\begin{itemize}
    \item \textbf{Analítica:} Proceso de detectar, interpretar y comunicar patrones significativos en los datos, así como usar herramientas para que toda la organización pueda realizar cualquier pregunta sobre cualquier información en todos los entornos y dispositivos posibles \cite{oracle-2021}.

    \item \textbf{Rendimiento:} Razón con unidad dimensional [MASA][PRODUCTO]/[ÁREA] útil para medir el beneficio producido en una zafra (p. ej. 1TA/HA 1 tonelada de azucar por hectárea).

    \item \textbf{Zafra:} Cosecha de la caña dulce (p. ej. zafra 2020-2021).
\end{itemize}

\subsubsection{Términos de Dominio Específico}

\begin{itemize}
    \item \textbf{Usuario:} Empresa agrícola en Honduras.

    \item \textbf{Usuario interno:} Empleado de la empresa agrícola.

    \item \textbf{Muestreo físico:} Muestra original que el usuario tomó en su suelo agrícola.

    \item \textbf{Muestreo virtual:} Resultado o salida del modelo que se ha de desarrollar en este estudio el cual tiene como entrada el muestreo físico.

    \item \textbf{Curar:} Transformar los datos originales mediante filtros de forma que se eliminen los todos aquellos datos innecesarios o corruptos para el análisis subyacente.
\end{itemize}

\printbibliography

\end{document}
