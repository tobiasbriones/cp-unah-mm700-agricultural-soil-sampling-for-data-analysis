% Copyright (c) 2021 Tobias Briones. All rights reserved.
% SPDX-License-Identifier: CC-BY-4.0
%
% This file is part of https://github.com/tobiasbriones/
% cp-unah-mm700-agricultural-soil-sampling-for-data-analysis
%
% This source code is licensed under the Creative Commons Attribution 4.0
% International License found in the LICENSE-CC file in the root directory of
% this source tree or at https://spdx.org/licenses/CC-BY-4.0

\section{Estimar el Tamaño de la Muestra}

Al estimar el tamaño de la muestra se tiene en cuenta otras definiciones como el margen de error que se considera tolerable. Las encuestas se pueden llevar a cabo para obtener estas condiciones correctamente. Para obtener resultados con respecto al tamaño de muestra preferible se sugiere \cite{lohr-2009}:

\begin{enumerate}
    \item Preguntárse: ¿Qué se espera de la muestra, y cuánta precisión se necesita?, ¿Qué consecuencias tienen los resultados del muestreo?, ¿Cuánto error es tolerable?.

    \item Encontrar una ecuación que relacione el tamaño de la muestra $n$ y las expectativas de la muestra.

    \item Estimar cualquier cantidad desconocida y resolver para $n$.

    \item Iterar estos pasos para obtener mejores valores de acuerdo a las expectativas. Si ni siquiera hay recursos disponibles para este proceso, entonces la investigación no se podrá llevar a cabo.
\end{enumerate}
