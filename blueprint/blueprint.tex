\documentclass{report}
\usepackage[utf8]{inputenc}
\usepackage[spanish]{babel}
\usepackage[letterpaper, portrait, margin=2cm]{geometry}
\usepackage{amsthm}
\usepackage{amsmath}
\usepackage{amssymb}
\usepackage{amsfonts}
\usepackage{hyperref}
\usepackage{csquotes}
\usepackage{mathtools}
\usepackage[
    backend=biber,
    style=alphabetic,
    sorting=ynt
]{biblatex}

\addbibresource{bibliography.bib}

\newtheorem{definition}{Definición}

\DeclarePairedDelimiter{\ceil}{\lceil}{\rceil}

\title{Anteproyecto de Investigación: Muestro de suelo agrícola para realizar análisis de datos prescriptivo y predictivo}
\author{
Tobias Briones\\\\
Anteproyecto de seminario de investigación presentado a la\\
Universidad Nacional Autónoma de Honduras de la carrera de\\
Licenciatura en Matemática
}
\date{Septiembre 2021}


\begin{document}

\maketitle

\tableofcontents


\section{Introducción}

Se ha tomado nota sobre la elaboración y estructura de \textit{Manual de Elaboración y Presentación de Tesis} \cite{universidad-san-carlos-2016}.

\textbf{Palabras clave:} diseño-de-muestreo, muestreo-estratificado, muestreo-de-suelo-agrícola, muestreo-para-análisis-de-datos, modelo-de-muestreo.


\section{Justificación}

El propósito de esta investigación es diseñar un resultado que permita modelar los suelos agrícolas de forma que se puedan obtener muestras representativas de estos y posteriormente desarrollar un proyecto con el sector agrícola integrando los resultados obtenidos. Se ha propuesto como objetivo emprender una investigación en este campo relacionado con el análisis de los suelos agrícolas para poder eventualmente publicar los resultados e integrar los modelos obtenidos en proyectos de analítica. Esto permitirá proveer un servicio/producto con un estado del arte en la región para los prospectos clientes de la empresa. Los beneficiados serán empresas del sector agrícola en Honduras al poder modelar sus suelos o lotes y así optimizar sus operaciones mediante técnicas de análisis de datos. Los recursos para realizar la extracción de datos en el suelo agrícola son muy limitados ya que la población es demasiado grande. Se debe poder entrenar los respectivos modelos de analítica a partir de los datos obtenidos del muestreo el cual debe ser suficientemente representativo y pequeño para poder llevar a cabo esta extracción de datos utilizando los recursos limitados que se tienen disponibles.

\bigbreak

Como limitantes en este proyecto se encuentran: la cota de tiempo para su realización consistente en un curso trimestral de seminario de investigación; llevar a cabo la recopilación física de datos o muestras. Debido a estas limitaciones se define el alcance del proyecto abajo.

\section{Alcance}

Dado las limitantes de esta investigación, se procederá a realizar las siguientes metas:

\begin{itemize}
    \item Obtener los modelos iniciales ya establecidos.
    
    \item Crear modelos ad hoc parar introducir definiciones de negocio. Esto con visión a que estos resultados deben de poder utilizarse en pro para software de analítica.
    
    \item Desarrollar gráficos, herramientas o simulaciones necesarias para la explicación de los fenómenos. En caso de requerir programación, se utilizará Python ya que se tratan de prototipos y se acopla perfectamente a este proyecto de investigación.
    
    \item Crear entrevistas/encuestas con personal del sector agrícola en caso de requerirlo.
\end{itemize}

\bigbreak

No son objetivos de este trabajo:

\begin{itemize}
    \item Desarrollar herramientas o aplicativos de grado de producción. Así como acoplar el tema central de la investigación a otros temas o proyectos de interés a fin de mantener resultados transparentes e independientes.
    
    \item Indagar más de lo necesario en aspectos particulares del sector agrícola a fin de incrementar la rigurosidad matemática que se desarrolle.
    
    \item Buscar resultados en la parte de ciencia de datos. El contexto de este trabajo consiste en obtener resultados para muestreo de lotes los cuales puedan ser utilizados en el campo de la ciencia de datos posteriormente.
\end{itemize}

\section{Planteamiento del problema}


\section{Hipótesis}

Todos los suelos agrícolas se pueden particionar como lotes y existe una relación de homogeneidad entre cada lote.

\section{Objetivos}


\subsection{Objetivo general}

Demostrar un proceso en el cual se pueda crear muestras representativas de lotes de zona agrícola tal que sean eficientes tanto de extraer como de obtener resultados de analítica a partir de ellas.

\subsection{Objetivos específicos}

\begin{itemize}
    \item Encontrar inductivamente modelos generales para analizar los suelos y lotes. 
    \item Diseñar un proceso para determinar un muestreo estratificado de lotes.
    \item Establecer un marco teórico estadístico para medir las variables en cuestión.
    \item Diseñar un modelo de selección de lotes por homogeneidad.
    \item Medir las relaciones para determinar cuan fina sera la partición.
    \item Concluir en un resultado final que pueda ser utilizado por software de analítica.
\end{itemize}


\subsection{Preguntas de investigación}

\begin{itemize}
    \item ¿Cómo particionar en lotes un suelo?
    \item ¿Qué relación existe entre cada lote?
    \item Si hay lotes relacionados, ¿qué métrica define al representante de cada subconjunto de la partición?
    \item ¿Se pueden crear particiones arbitrariamente finas?
    
    \item ¿Qué variables y modelos pertenecen a cada lote?
    \item ¿Qué relación existe entre los recursos de tiempo y dinero contra el tamaño de la muestra/partición?
\end{itemize}

\subsection{Población}


\subsection{Muestra}


\subsection{Metodología}


\section{Marco teórico}

Entre los tipos de muestreo que se han encontrado útiles para medir los suelos agrícolas se tienen:

\begin{itemize}
    \item \textbf{Muestreo Simple Aleatorio MSA:} Consiste en tomar $n$ puntos aleatorios de la población. Estos puntos deben de tener la misma probabilidad de ser escogidos para que la muestra sea representativa y se toman de forma "mezclada". Por ejemplo, la sangre esta "mezclada" en el cuerpo humano, por lo que al tomar una muestra de solo una pizca basta para hacer los análisis ya que esa pizca obtenida es igual que todas las demás.
    
    \item \textbf{Muestreo Simple Estratificado:} Esta es una técnica de muestreo que será muy útil en el modelado de los suelos agrícolas. La población se particiona en subconjuntos de diferentes tipos y homogéneos de forma que se puede realizar un MSA en cada subconjunto homogéneo de la partición. Por ejemplo, los lotes se pueden particionar como % TODO (tipos de lotes)
    
    \item \textbf{Muestreo sistemático:} En este muestreo se toma un punto aleatorio y se mide por cada $n$-ésima unidad de forma que se lleva un espaciado constante a partir del punto inicial. Por ejemplo, este tipo de muestreo es utilizado para medir el suelo cuando es rectangular a lo largo de su perímetro o también cuando es de forma irregular.
\end{itemize}

\subsection{Marco para muestreo}

El marco para muestreo \cite{lohr-2009} consiste en definir el espacio de la población (el universo) de donde se obtendrán las posibles muestras (subconjuntos) y estas muestras contienen las unidades que serán seleccionadas. Se nota que cada muestra tiene una probabilidad de ser escogida y para cada muestra, cada unidad tiene también una probabilidad para ser escogida.

\begin{definition}[Universo]
    El \textbf{Universo} o \textbf{Población finita} de $N$ unidades es el conjunto índice
    $$
    U = \{ 1, 2, ..., N \}
    $$
    Donde $N \in \mathbb{N}$ es el tamaño de la población.
\end{definition}

\begin{definition}[Muestra]
    Sea $U$ el conjunto universo. Un conjunto $S$ es una muestra para $U$ si $S \subseteq U$.
\end{definition}

\begin{definition}[Probabilidad de una muestra]
    Si $S$ es una muestra. $S$ tiene una probabilidad de ser escogida de $P(S)$.
\end{definition}

Notar que la probabilidad de todas las muestras de ser escogida es $1$. Esto es, $\forall S_i \in D \implies \sum_{i=1}^{N} P(S_i) = 1$, donde el conjunto $D$ es el diseño escogido, esto es, una colección de subconjuntos (muestras) de $U$.

\bigbreak

Según las definiciones de arriba, cada muestra $S_i$ tiene probabilidad $P(S_i)$ de ser seleccionada. Ahora, cada unidad tiene una probabilidad de terminar siendo seleccionada si pertenece a una de las muestras que se seleccionó.

\begin{definition}[Probabilidad de una Unidad]
    La probabilidad de que una unidad $x$ sea seleccionada en una colección de muestras $D = \{ S_1, S_2, ..., S_n \}$ se define como
    
    $$
    \pi_x = P(\text{unidad $x$ está en la muestra}) = \sum_{S_i \in \{ S \in D | x \in S \}} P(S_i)
    $$
\end{definition}

Es decir, para calcular la probabilidad de que la unidad $x$ sea seleccionada, se hace la suma de las probabilidades de todas las muestras que contienen a $x$.

Con respecto a los intervalos de confianza, se deberá repetir muchas veces el muestreo para por determinar su confianza. Son útiles en el caso no-ideal donde no se conoce toda la población. Los intervalos de confianza se construyen como \cite{the-pennsylvania-state-university-no-date}:

\begin{definition}[Intervalo de confianza]
    Un \textbf{intervalo de confianza} es un rango que se calcula usando estadística para estimar un parámetro desconocido de la población con un nivel determinado de confianza. 
\end{definition}

\begin{definition}[Estimación puntual]
    La \textbf{estimación puntual} es una muestra estadística que sirve como los mejores estimados para un parámetro de la población.
\end{definition}

\begin{definition}[Margen de error]
    El \textbf{margen de error} de un intervalo de confianza es la mitad de su ancho.
\end{definition}




\printbibliography

\end{document}
