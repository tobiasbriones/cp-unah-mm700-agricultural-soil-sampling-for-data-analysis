\documentclass{report}
\usepackage[utf8]{inputenc}
\usepackage[spanish]{babel}
\usepackage[letterpaper, portrait, margin=2cm]{geometry}
\usepackage{amsthm}
\usepackage{hyperref}
\usepackage{csquotes}
\usepackage[
    backend=biber,
    style=alphabetic,
    sorting=ynt
]{biblatex}

\addbibresource{bibliography.bib}

\newtheorem{definition}{Definición}

\title{Anteproyecto de Investigación: Muestro de suelo agrícola para realizar análisis de datos prescriptivo y predictivo}
\author{
Tobias Briones\\\\
Anteproyecto de seminario de investigación presentado a la\\
Universidad Nacional Autónoma de Honduras de la carrera de\\
Licenciatura en Matemática
}
\date{Septiembre 2021}


\begin{document}

\maketitle

\tableofcontents


\section{Introducción}


\section{Justificación}

El propósito de esta investigación es diseñar un resultado que permita modelar los suelos agrícolas de forma que se puedan obtener muestras representativas de estos y posteriormente desarrollar un proyecto con el sector agrícola integrando los resultados obtenidos. Se ha propuesto como objetivo emprender una investigación en este campo relacionado con el análisis de los suelos agrícolas para poder eventualmente publicar los resultados e integrar los modelos obtenidos en proyectos de analítica. Esto permitirá proveer un servicio/producto con un estado del arte en la región para los prospectos clientes de la empresa. Los beneficiados serán empresas del sector agrícola en Honduras al poder modelar sus suelos o lotes y así optimizar sus operaciones mediante técnicas de análisis de datos. Los recursos para realizar la extracción de datos en el suelo agrícola son muy limitados ya que la población es demasiado grande. Se debe poder entrenar los respectivos modelos de analítica a partir de los datos obtenidos del muestreo el cual debe ser suficientemente representativo y pequeño para poder llevar a cabo esta extracción de datos utilizando los recursos limitados que se tienen disponibles.

\bigbreak

Como limitantes en este proyecto se encuentran: la cota de tiempo para su realización consistente en un curso trimestral de seminario de investigación; llevar a cabo la recopilación física de datos o muestras. Debido a estas limitaciones se define el alcance del proyecto abajo.

\section{Alcance}

Dado las limitantes de esta investigación, se procederá a realizar las siguientes metas:

\begin{itemize}
    \item Obtener los modelos iniciales ya establecidos.
    
    \item Crear modelos ad hoc parar introducir definiciones de negocio. Esto con visión a que estos resultados deben de poder utilizarse en pro para software de analítica.
    
    \item Desarrollar gráficos, herramientas o simulaciones necesarias para la explicación de los fenómenos. En caso de requerir programación, se utilizará Python ya que se tratan de prototipos y se acopla perfectamente a este proyecto de investigación.
    
    \item Crear entrevistas/encuestas con personal del sector agrícola en caso de requerirlo.
\end{itemize}

\bigbreak

No son objetivos de este trabajo:

\begin{itemize}
    \item Desarrollar herramientas o aplicativos de grado de producción. Así como acoplar el tema central de la investigación a otros temas o proyectos de interés a fin de mantener resultados transparentes e independientes.
    
    \item Indagar más de lo necesario en aspectos particulares del sector agrícola a fin de incrementar la rigurosidad matemática que se desarrolle.
    
    \item Buscar resultados en la parte de ciencia de datos. El contexto de este trabajo consiste en obtener resultados para muestreo de lotes los cuales puedan ser utilizados en el campo de la ciencia de datos posteriormente.
\end{itemize}

\section{Planteamiento del problema}


\section{Hipótesis}



\section{Objetivos}


\subsection{Objetivos generales}


\subsection{Objetivos específicos}



\subsection{Preguntas de investigación}

\begin{itemize}
    \item ¿Cómo particionar en lotes un suelo?
    \item ¿Qué relación existe entre cada lote?
    \item Si hay lotes relacionados, ¿qué métrica define al representante de cada subconjunto de la partición?
    \item ¿Se pueden crear particiones arbitrariamente finas?
    
    \item ¿Qué variables y modelos pertenecen a cada lote?
    \item ¿Qué relación existe entre los recursos de tiempo y dinero contra el tamaño de la muestra/partición?
\end{itemize}

\subsection{Población}


\subsection{Muestra}


\subsection{Metodología}


\section{Revisión de literatura}

La bibliografía de este proyecto de investigación consiste en:

\begin{itemize}
    \item Centro regional Misiones Estación Experimental Agropecuaria Cerro Azul, Sosa, D., \& Alvarenga, F. (n.d.). Técnicas de toma y remisión de muestras de suelos. Instituto Nacional Tecnología Agropecuaria. Retrieved September 2021, from \url{https://inta.gob.ar/sites/default/files/script-tmp-tcnicas_de_toma_y_remisin_de_muestras_de_suelos.pdf}
    
    \item Guía para muestreo de suelos. (2014). Gob.Pe Ministerio Del Ambiente. \url{https://www.minam.gob.pe/wp-content/uploads/2014/04/GUIA-MUESTREO-SUELO_MINAM1.pdf}
    
    \item Instituto Nacional de Ecología. (n.d.). INECC Instituto Nacional de Ecología y Cambio Climático. Retrieved September 2021, from \url{http://www2.inecc.gob.mx/publicaciones2/libros/459/cap3.html}
    
    \item Lassaga, S. \& Instituto Nacional de Innovación y Transferencia en Tecnología Agropecuaria. (2011). MUESTREO Y ANÁLISIS DE SUELOS PARA DIAGNÓSTICO DE FERTILIDAD. Ministerio de Agricultura y Ganadería de Costa Rica. \url{http://www.mag.go.cr/bibliotecavirtual/P33-9965.pdf}
    
    \item Lohr, S. (2005). Muestro: Diseño y Análisis. International Thomson Editores.
    
    \item Lohr, S. L. (2009). Sampling: Design and Analysis (Advanced Series) (2nd ed.). Cengage Learning.
    
    \item Organización de las naciones unidas para la agricultura y la alimentación. (1990). Métodos de muestreo para las encuestas agrícolas. Food and Agriculture Organization of the United Nations. \url{http://www.fao.org/3/ca5865es/CA5865ES.pdf}
\end{itemize}

La bibliografía respecto a la elaboración de este trabajo consisten en el \textit{Manual de Elaboración y Presentación de Tesis} \cite{universidad-san-carlos-2016}.

\printbibliography

\end{document}
