\documentclass{report}
\usepackage[utf8]{inputenc}
\usepackage[spanish]{babel}
\usepackage[letterpaper, portrait, margin=2cm]{geometry}
\usepackage[style=ieee]{biblatex}
\usepackage{amsthm}
\usepackage{amsmath}
\usepackage{amssymb}
\usepackage{amsfonts}
\usepackage{hyperref}
\usepackage{csquotes}
\usepackage{mathtools}
\usepackage{graphicx}
\usepackage{float}
\usepackage{array}
\usepackage{dirtytalk}
\usepackage[table,xcdraw]{xcolor}

\addbibresource{bibliography.bib}

\newtheorem{definition}{Definición}

\DeclarePairedDelimiter{\ceil}{\lceil}{\rceil}

\title{Planteamiento del Problema de Investigación: Modelo de muestreo virtual de suelos agrícolas en Honduras para analítica en base al muestreo aleatorio simple y estratificado}
\author{Tobias Briones \bigbreak tobias.briones@unah.hn}
\date{Septiembre 2021}

\begin{document}

\makeatletter
    \begin{titlepage}
        \begin{center}
            \includegraphics[width=0.3\linewidth]{ref/logo-unah.png}\\[4ex]
            {\huge \bfseries \@title 
            \vspace{1cm}}\\[2ex]
            {\LARGE \@author}\\[50ex] 
            
            {\large
            Anteproyecto de seminario de investigación presentado a la\\
            Universidad Nacional Autónoma de Honduras de la carrera de\\
            Licenciatura en Matemática
            }\\[2ex]
            
            {\large \@date}
        \end{center}
    \end{titlepage}
\makeatother
\thispagestyle{empty}
\newpage

\thispagestyle{empty}
\tableofcontents
\newpage

\chapter{Introducción}

En la ciencia de datos \footnote{La ciencia de datos combina múltiples campos, como las estadísticas, los métodos científicos, la inteligencia artificial (IA) y el análisis de datos para extraer el valor de los datos \cite{oracle-data-science-2021}.} se desarrollan procesos complicados para obtener valor a partir de los datos crudos de los usuarios de forma que se puedan entrenar modelos para hacer predicciones, prescripciones e investigación de los problemas de las empresas en una base de caso por caso. Uno de los primeros pasos en el proceso consiste en obtener datos existentes y que sufran una transformación para que se filtren o curen y por tanto se obtengan datos relevantes al estudio. Así también como determinar la clasificación correcta de los datos, determinar variables de estudio, desnormalizar bases de datos, encontrar patrones y establecer  interpretaciones \cite{university-of-wisconsin-data-science-2021}. Debido a estos retos técnicos, el analista de datos \footnote{El analista de datos actúa como guardián de los datos de una organización para que las partes interesadas puedan comprender los datos y usarlos para tomar decisiones comerciales estratégicas \cite{eastwood-data-analyst-2021}.} debe de contar como entrada con todos los datos e historiales de la empresa que se va a analizar en ese dominio.

\bigbreak

A fin de obtener los datos de fincas, granjas o terreno agrícola es requerido muchas veces realizar muestreo de suelo. Algunos de los casos de uso son diagnóstico de fertilidad \cite{lassaga-2011} y análisis de contaminantes \cite{gobpe-ministerio-del-ambiente-2014}. Es determinante construir resultados a partir de estos datos para dar resultados en base a zafra y sus correspondiente diagnóstico de optimalidad con respecto a rendimiento.

\bigbreak

Existen varios enfoques de muestreo de una población, la mayor parte de las veces se basan en un muestreo aleatorio simple. Para poder diseñar un muestreo para un caso en particular es requerido llevar a cabo encuestas puntuales al personal del área agrícola a fin de conocer el terreno y sus clasificaciones \cite{organizacion-de-las-naciones-unidas-para-la-agricultura-y-la-alimentacion-1990}.

\bigbreak

Para realizar un diseño de muestreo es necesario indagar en la teoría de muestreo elemental, recurrir a la ciencia de datos en las primeras etapas del análisis de estos casos de uso (transformaciones de datos que se obtienen al inicio del proceso de analítica) y definitivamente a las herramientas que llevarán a cabo el artefacto de investigación consistentes en las bibliotecas de ciencia de datos de Python \footnote{Python es un lenguaje de programación de propósito general de alto nivel e interpretado \cite{wikipedia-python-2021}.} \cite{grus-2015} \cite{geopandas-developers-2021}. Por último, sin dejar de lado el otro aspecto complementario para diseñar tal muestreo, se deberá tener en consideración algunas condicionantes del área local en Honduras de forma que el artefacto de investigación sea desplegable en los casos de uso que conciernen más a esta área geográfica \cite{fao-2004}.

\section{Justificación}

El propósito de esta investigación es diseñar un proceso eficiente que permita modelar los suelos agrícolas de forma que se puedan obtener muestras representativas de estos y posteriormente ser utilizables para que permitan desarrollar proyectos con el sector agrícola integrando los resultados obtenidos de esta investigación. Se ha propuesto como objetivo emprender una investigación en este campo relacionado con el análisis de los suelos agrícolas para poder integrar los modelos obtenidos en proyectos de analítica. Esto se espera que permita a las empresas de analítica \footnote{La analítica reúne la teoría y la práctica para identificar y comunicar conocimientos basados en datos que permiten a los gerentes, partes interesadas y otros ejecutivos de una organización tomar decisiones más informadas \cite{eastwood-data-analyst-2021}.} poder proveer servicios/productos con un estado del arte en la región para sus prospectos clientes o usuarios los cuales consisten en empresas agrícolas en Honduras que deberían automatizar y llevar a cabo el análisis de suelos de forma más moderna que como lo hacen actualmente o, en el peor de los casos, no hacen ni son capaces de entender ningún análisis ni optimización u automatización en absoluto. Los beneficiados serán primero, empresas del sector agrícola en Honduras al poder darle uso a los históricos actuales y futuros de sus datos a fin de modelar sus suelos o lotes y así optimizar sus operaciones mediante técnicas de análisis de datos, lo cual ya se puede hacer. La diferencia es que, por el otro lado, se tiene como beneficiado segundo a los proveedores de servicios de analítica al poder entrenar y desplegar los modelos de ciencia de datos de forma \textit{más eficiente} si se utiliza un muestreo virtual propuesto en esta investigación. Los recursos para realizar la extracción de datos en el suelo agrícola son muy limitados ya que la población es demasiado grande. Se debe poder entrenar los respectivos modelos de analítica a partir de los datos obtenidos del muestreo el cual debe ser suficientemente representativo y pequeño para poder llevar a cabo esta extracción de datos utilizando los recursos limitados que se tienen disponibles. En síntesis, las empresas agrícolas tienen ya sus históricos, las empresas de analítica en teoría pueden optimizar en base a esos datos (muestra física) pero es aún posible tener mayor ganancia para ambos sector agrícola y sector de analítica al hacer que el análisis sea más eficiente al reducir esa entrada de datos mediante un segundo muestreo que será virtual.

\bigbreak

Según el \textit{Tratado Internacional de  Recursos Fito Genéticos  para la Alimentación y la Agricultura  TIRFAA} del \textit{Gobierno de la República de Honduras $\mid$ Secretaría de Agricultura y Ganadería} (2019) \cite{santacreo-2019} \say{En la actualidad no se cuenta con un inventario exhaustivo a nivel nacional de las variedades tradicionales cultivadas en fincas, principalmente de pequeños productores (campesinos), los parientes y especies silvestres utilizadas para la producción de alimentos}. Lo cual sugiere que es importante que los datos sobre las variedades de diversos productores en el país deberían ser explotados para tener manejo sobre el valor que puede agregar la ciencia de datos a esas brechas de inventario existentes.

\bigbreak

Como limitantes en este proyecto se encuentran: la cota de tiempo para su realización consistente en un curso trimestral de seminario de investigación. Por otra parte, llevar a cabo la recopilación física de datos o muestras no es necesario ya que el muestreo físico va por parte del sector agrícola; siendo así, una potencial limitante es poder obtener un conjunto de datos apropiado de acceso abierto para poder hacer la demostración del procedimiento en cuestión. Debido a estas limitaciones se define el alcance del proyecto abajo.

\section{Audiencia}

Esta investigación involucra a todos aquellos interesados en la industria de analítica/ciencia de datos y de agricultura en Honduras. El tema se define en base a la etapa intermedia entre la empresa agrícola que cuenta con datos de muestreo físico y la empresa de analítica/ciencia de datos que ha de entrenar modelos en base al propuesto muestreo virtual.

\section{Alcance}

Dado las limitantes de esta investigación, se procederá a realizar las siguientes metas:

\begin{itemize}
    \item Obtener los modelos iniciales ya establecidos para dichos conjunto de datos, esto a fin de contar con un marco teórico con base al problema.
    
    \item Crear modelos ad hoc parar introducir definiciones de negocio. Esto con visión a que estos resultados deben de poder utilizarse en pro para software de analítica.
    
    \item Desarrollar gráficos, herramientas o simulaciones necesarias para la explicación de los fenómenos. En caso de requerir programación, se utilizará Python ya que se tratan de prototipos y se acopla perfectamente a este proyecto de investigación junto con el soporte en primera clase y estandarización en bibliotecas que tiene en el momento.
    
    \item Crear entrevistas/encuestas con personal del sector agrícola en caso de requerirlo.
    
    \item Dejar un modelo estadístico/computacional como resultado que permita ser implementado y extendido en producción de acuerdo a las necesidades de cada proyecto real en una base de caso por caso.
\end{itemize}

\chapter{Planteamiento del Problema}

\section{Hipótesis}

Todos los suelos agrícolas se pueden particionar como lotes y existe una relación de homogeneidad entre cada lote.

\section{Objetivos}

\subsection{Objetivo General}

Modelar un proceso en el cual se obtengan muestras de lotes de suelo agrícola dado las siguientes condiciones:

\begin{itemize}
    \item Asumir un conjunto de datos ya muestreados físicamente (por parte de la empresa agrícola) como entrada del proceso.
    
    \item El muestreo es estratificado basado en el MAS.
    
    \item El proceso es de utilidad para las etapas iniciales de analítica.
    
    \item La salida del proceso (el muestreo) permite posteriormente entrenar modelos de analítica más eficientemente al minimizar sus entradas con respecto a la muestra original.
\end{itemize}

\subsection{Objetivos Específicos}

\begin{itemize}
    \item Encontrar inductivamente modelos generales para analizar los suelos y lotes. 
    
    \item Diseñar un proceso para determinar un muestreo estratificado de lotes.
    
    \item Identificar preguntas clave para obtener estratos a partir de encuestas a los usuarios.
    
    \item Establecer un marco teórico estadístico para medir las variables en cuestión.
    
    \item Diseñar un modelo de selección de lotes por homogeneidad.
    
    \item Medir las relaciones para determinar cuan fina será la partición.
    
    \item Concluir en un resultado final que pueda ser utilizado por software de analítica.
    
    \item Proveer las herramientas de análisis de datos para visualizar y configurar la entrada y salida del proceso.
\end{itemize}


\section{Preguntas de Investigación}

\begin{itemize}
    \item ¿Qué tan curados son los datos (muestra física) que manejan los usuarios para hacer análisis de rendimiento?
    
    \item ¿Qué tiempo y espacio de complejidad tienen los modelos de analítica para optimización de rendimiento?
    
    \item ¿Qué medio (correo electrónico, papel, app, etc.) electrónico o físico es el más eficiente para obtener encuestas agrícolas de usuarios en Honduras?
    
    \item ¿Qué usuarios internos son los más claves para preguntas específicas?
    
    \item ¿Qué usuarios internos son los más claves para preguntas generales?
    
    \item ¿Qué modelo o modelos estadísticos se utilizan para generar muestreos estratificados a partir de un conjunto de datos?
    
    \item ¿Cuáles son los estratos de suelos agrícolas en Honduras?
    
    \item ¿Cómo es la entrada (el muestreo virtual) de los procesos de analítica para análisis de suelos agrícola en términos de transformaciones (filtrado, mapeo, etc.), estructura (desnormalizado, relacional, etc.) con respecto al muestreo físico?
    
    \item ¿Qué información geográfica manejan los usuarios sobre sus fincas, lotes o suelos agrícolas?
    
    \item ¿El muestreo físico histórico está relacionado con el actual o evoluciona en función de las zafras?
    
    \item ¿Qué tipo de cultivo/cosecha/variedad se maneja en las fincas de Honduras? ¿Qué relaciones (principalmente biológicas) hay entre ellas?
\end{itemize}

\section{Metodología}

La metodología para llevar a cabo este proyecto es de carácter mixta, es decir, cuantitativa y cualitativa. En el análisis cuantitativo se tienen los modelos estadísticos y de ciencia de datos que son computados por las bibliotecas de Python. Por otra parte, los datos que se analizan en agricultura y recolectados mediante encuestas y filtración de datos crudos (iniciales) tienen etiquetas que dan valor de negocio (zafra, cosecha, rendimiento, etc.).

\bigbreak

Con respecto al enfoque cuantitativo se tiene que este estudio es de carácter deductivo al aplicar métodos estadísticos y de teoría de muestreo además de reducir el uso de la computación a dominio específico para dicho problema de investigación. Por otra parte, se deberá de generalizar el modelo a plantear a partir de la información agrícola del área local y sus necesidades. En tanto a encuestas, esto conduce a utilizar una parte del método cualitativo para, por ejemplo, obtener el tipo de cosechas que se producen en la región y que serán capturadas por el muestreo correspondiente. La parte cualitativa ha de permitir poder establecer la homogeneidad entre los estratos para el muestreo estratificado.

\subsection{Revisión de Bibliografía}

Se consultó con diversas fuentes de información para obtener un marco teórico que permitió desprender todos los detalles necesarios para el artefacto de investigación incluyendo definiciones y procesos de agricultura en organizaciones pertinentes al tema como la FAO.org, cálculos estadísticos de libros de muestreo y documentación sobre las potenciales herramientas en Python para este tipo de aplicaciones.

\chapter{Anexos}

Se ha tomado nota en pequeña parte sobre la elaboración y estructura de \textit{Manual de Elaboración y Presentación de Tesis} \cite{universidad-san-carlos-2016}, \textit{Metodología de la investigación} \cite{collado-2014}, \textit{Introducción a la metodología de la investigación científica} \cite{cabezas-2018} para esta propuesta de investigación.

\section{Glosario}

\begin{itemize}
    \item \textbf{Analítica:} Proceso de detectar, interpretar y comunicar patrones significativos en los datos, así como usar herramientas para que toda la organización pueda realizar cualquier pregunta sobre cualquier información en todos los entornos y dispositivos posibles \cite{oracle-2021}.
    
    \item \textbf{Rendimiento:} Razón con unidad dimensional [MASA][PRODUCTO]/[ÁREA] útil para medir el beneficio producido en una zafra (p. ej. 1TA/HA 1 tonelada de azucar por hectárea).
    
    \item \textbf{Zafra:} Cosecha de la caña dulce (p. ej. zafra 2020-2021).
    
    \item \textbf{Variedad vegetal:} Grupo de plantas definido con mayor precisión, seleccionado dentro de una especie, que presentan una serie de características comunes \cite{union-internacional-para-la-proteccion-de-las-obtenciones-vegatales-upov-2021}.
\end{itemize}

\subsubsection{Términos de dominio específico}

\begin{itemize}
    \item \textbf{Usuario:} Empresa agrícola en Honduras.
    
    \item \textbf{Usuario interno:} Empleado de la empresa agrícola.
    
    \item \textbf{Muestreo físico:} Muestra original que el usuario tomó en su suelo agrícola.
    
    \item \textbf{Muestreo virtual:} Resultado o salida del modelo que se ha de desarrollar en este estudio el cual tiene como entrada el muestreo físico.
    
    \item \textbf{Curar:} Transformar los datos originales mediante filtros de forma que se eliminen los todos aquellos datos innecesarios o corruptos para el análisis subyacente.
\end{itemize}

\printbibliography

\end{document}
