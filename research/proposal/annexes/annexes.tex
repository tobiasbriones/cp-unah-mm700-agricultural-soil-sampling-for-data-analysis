% Copyright (c) 2021 Tobias Briones. All rights reserved.
% SPDX-License-Identifier: CC-BY-4.0
%
% This file is part of https://github.com/tobiasbriones/
% cp-unah-mm700-agricultural-soil-sampling-for-data-analysis
%
% This source code is licensed under the Creative Commons Attribution 4.0
% International License found in the LICENSE-CC file in the root directory of
% this source tree or at https://spdx.org/licenses/CC-BY-4.0

Se ha tomado nota en pequeña parte sobre la elaboración y estructura de \textit{Manual de Elaboración y Presentación de Tesis} \cite{universidad-san-carlos-2016}, \textit{Metodología de la investigación} \cite{collado-2014}, \textit{Introducción a la metodología de la investigación científica} \cite{cabezas-2018} para esta propuesta de investigación.

\section{Glosario}

\begin{itemize}
    \item \textbf{Analítica:} Proceso de detectar, interpretar y comunicar patrones significativos en los datos, así como usar herramientas para que toda la organización pueda realizar cualquier pregunta sobre cualquier información en todos los entornos y dispositivos posibles \cite{oracle-2021}.

    \item \textbf{Rendimiento:} Razón con unidad dimensional [MASA][PRODUCTO]/[ÁREA] útil para medir el beneficio producido en una zafra (p. ej. 1TA/HA 1 tonelada de azucar por hectárea).

    \item \textbf{Zafra:} Cosecha de la caña dulce (p. ej. zafra 2020-2021).

    \item \textbf{Variedad vegetal:} Grupo de plantas definido con mayor precisión, seleccionado dentro de una especie, que presentan una serie de características comunes \cite{union-internacional-para-la-proteccion-de-las-obtenciones-vegatales-upov-2021}.
\end{itemize}

\subsubsection{Términos de Dominio Específico}

\begin{itemize}
    \item \textbf{Usuario:} Empresa agrícola en Honduras.

    \item \textbf{Usuario interno:} Empleado de la empresa agrícola.

    \item \textbf{Muestreo físico:} Muestra original que el usuario tomó en su suelo agrícola.

    \item \textbf{Muestreo virtual:} Resultado o salida del modelo que se ha de desarrollar en este estudio el cual tiene como entrada el muestreo físico.

    \item \textbf{Curar:} Transformar los datos originales mediante filtros de forma que se eliminen los todos aquellos datos innecesarios o corruptos para el análisis subyacente.
\end{itemize}
