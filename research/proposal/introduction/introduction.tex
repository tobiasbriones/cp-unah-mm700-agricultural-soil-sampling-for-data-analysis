% Copyright (c) 2021 Tobias Briones. All rights reserved.
% SPDX-License-Identifier: CC-BY-4.0
%
% This file is part of https://github.com/tobiasbriones/
% cp-unah-mm700-agricultural-soil-sampling-for-data-analysis
%
% This source code is licensed under the Creative Commons Attribution 4.0
% International License found in the LICENSE-CC file in the root directory of
% this source tree or at https://spdx.org/licenses/CC-BY-4.0

En la ciencia de datos \footnote{La ciencia de datos combina múltiples campos, como las estadísticas, los métodos científicos, la inteligencia artificial (IA) y el análisis de datos para extraer el valor de los datos \cite{oracle-data-science-2021}.} se desarrollan procesos complicados para obtener valor a partir de los datos crudos de los usuarios de forma que se puedan entrenar modelos para hacer predicciones, prescripciones e investigación de los problemas de las empresas en una base de caso por caso. Uno de los primeros pasos en el proceso consiste en obtener datos existentes y que sufran una transformación para que se filtren o curen y por tanto se obtengan datos relevantes al estudio. Así también como determinar la clasificación correcta de los datos, determinar variables de estudio, desnormalizar bases de datos, encontrar patrones y establecer  interpretaciones \cite{university-of-wisconsin-data-science-2021}. Debido a estos retos técnicos, el analista de datos \footnote{El analista de datos actúa como guardián de los datos de una organización para que las partes interesadas puedan comprender los datos y usarlos para tomar decisiones comerciales estratégicas \cite{eastwood-data-analyst-2021}.} debe de contar como entrada con todos los datos e historiales de la empresa que se va a analizar en ese dominio.

\bigbreak

A fin de obtener los datos de fincas, granjas o terreno agrícola es requerido muchas veces realizar muestreo de suelo. Algunos de los casos de uso son diagnóstico de fertilidad \cite{lassaga-2011} y análisis de contaminantes \cite{gobpe-ministerio-del-ambiente-2014}. Es determinante construir resultados a partir de estos datos para dar resultados en base a zafra y sus correspondiente diagnóstico de optimalidad con respecto a rendimiento.

\bigbreak

Existen varios enfoques de muestreo de una población, la mayor parte de las veces se basan en un muestreo aleatorio simple. Para poder diseñar un muestreo para un caso en particular es requerido llevar a cabo encuestas puntuales al personal del área agrícola a fin de conocer el terreno y sus clasificaciones \cite{organizacion-de-las-naciones-unidas-para-la-agricultura-y-la-alimentacion-1990}.

\bigbreak

Para realizar un diseño de muestreo es necesario indagar en la teoría de muestreo elemental, recurrir a la ciencia de datos en las primeras etapas del análisis de estos casos de uso (transformaciones de datos que se obtienen al inicio del proceso de analítica) y definitivamente a las herramientas que llevarán a cabo el artefacto de investigación consistentes en las bibliotecas de ciencia de datos de Python \footnote{Python es un lenguaje de programación de propósito general de alto nivel e interpretado \cite{wikipedia-python-2021}.} \cite{grus-2015} \cite{geopandas-developers-2021}. Por último, sin dejar de lado el otro aspecto complementario para diseñar tal muestreo, se deberá tener en consideración algunas condicionantes del área local en Honduras de forma que el artefacto de investigación sea desplegable en los casos de uso que conciernen más a esta área geográfica \cite{fao-2004}.
