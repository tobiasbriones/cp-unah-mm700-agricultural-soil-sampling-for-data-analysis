% Copyright (c) 2021 Tobias Briones. All rights reserved.
% SPDX-License-Identifier: CC-BY-4.0
%
% This file is part of https://github.com/tobiasbriones/
% cp-unah-mm700-agricultural-soil-sampling-for-data-analysis
%
% This source code is licensed under the Creative Commons Attribution 4.0
% International License found in the LICENSE-CC file in the root directory of
% this source tree or at https://spdx.org/licenses/CC-BY-4.0

\subsection{Justificación}

El propósito de esta investigación es diseñar un proceso eficiente que permita modelar los suelos agrícolas de forma que se puedan obtener muestras representativas de estos y posteriormente ser utilizables para que permitan desarrollar proyectos con el sector agrícola integrando los resultados obtenidos de esta investigación. Se ha propuesto como objetivo emprender una investigación en este campo relacionado con el análisis de los suelos agrícolas para poder integrar los modelos obtenidos en proyectos de analítica. Esto se espera que permita a las empresas de analítica \footnote{La analítica reúne la teoría y la práctica para identificar y comunicar conocimientos basados en datos que permiten a los gerentes, partes interesadas y otros ejecutivos de una organización tomar decisiones más informadas \cite{eastwood-data-analyst-2021}.} poder proveer servicios/productos con un estado del arte en la región para sus prospectos clientes o usuarios los cuales consisten en empresas agrícolas en Honduras que deberían automatizar y llevar a cabo el análisis de suelos de forma más moderna que como lo hacen actualmente o, en el peor de los casos, no hacen ni son capaces de entender ningún análisis ni optimización u automatización en absoluto. Los beneficiados serán primero, empresas del sector agrícola en Honduras al poder darle uso a los históricos actuales y futuros de sus datos a fin de modelar sus suelos o lotes y así optimizar sus operaciones mediante técnicas de análisis de datos, lo cual ya se puede hacer. La diferencia es que, por el otro lado, se tiene como beneficiado segundo a los proveedores de servicios de analítica al poder entrenar y desplegar los modelos de ciencia de datos de forma \textit{más eficiente} si se utiliza un muestreo virtual propuesto en esta investigación. Los recursos para realizar la extracción de datos en el suelo agrícola son muy limitados ya que la población es demasiado grande. Se debe poder entrenar los respectivos modelos de analítica a partir de los datos obtenidos del muestreo el cual debe ser suficientemente representativo y pequeño para poder llevar a cabo esta extracción de datos utilizando los recursos limitados que se tienen disponibles. En síntesis, las empresas agrícolas tienen ya sus históricos, las empresas de analítica en teoría pueden optimizar en base a esos datos (muestra física) pero es aún posible tener mayor ganancia para ambos sector agrícola y sector de analítica al hacer que el análisis sea más eficiente al reducir esa entrada de datos mediante un segundo muestreo que será virtual.

\bigbreak

Según el \textit{Tratado Internacional de  Recursos Fito Genéticos  para la Alimentación y la Agricultura  TIRFAA} del \textit{Gobierno de la República de Honduras $\mid$ Secretaría de Agricultura y Ganadería} (2019) \cite{santacreo-2019} \say{En la actualidad no se cuenta con un inventario exhaustivo a nivel nacional de las variedades tradicionales cultivadas en fincas, principalmente de pequeños productores (campesinos), los parientes y especies silvestres utilizadas para la producción de alimentos}. Lo cual sugiere que es importante que los datos sobre las variedades de diversos productores en el país deberían ser explotados para tener manejo sobre el valor que puede agregar la ciencia de datos a esas brechas de inventario existentes.

\bigbreak

Como limitantes en este proyecto se encuentran: la cota de tiempo para su realización consistente en un curso trimestral de seminario de investigación. Por otra parte, llevar a cabo la recopilación física de datos o muestras no es necesario ya que el muestreo físico va por parte del sector agrícola; siendo así, una potencial limitante es poder obtener un conjunto de datos apropiado de acceso abierto para poder hacer la demostración del procedimiento en cuestión. Debido a estas limitaciones se define el alcance del proyecto abajo.
