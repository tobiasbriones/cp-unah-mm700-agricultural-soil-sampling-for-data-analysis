% Copyright (c) 2021 Tobias Briones. All rights reserved.
% SPDX-License-Identifier: CC-BY-4.0
%
% This file is part of https://github.com/tobiasbriones/
% cp-unah-mm700-agricultural-soil-sampling-for-data-analysis
%
% This source code is licensed under the Creative Commons Attribution 4.0
% International License found in the LICENSE-CC file in the root directory of
% this source tree or at https://spdx.org/licenses/CC-BY-4.0

\subsection{Preguntas de Investigación}

\begin{itemize}
    \item ¿Qué tan curados son los datos (muestra física) que manejan los usuarios para hacer análisis de rendimiento?

    \item ¿Qué tiempo y espacio de complejidad tienen los modelos de analítica para optimización de rendimiento?

    \item ¿Qué medio (correo electrónico, papel, app, etc.) electrónico o físico es el más eficiente para obtener encuestas agrícolas de usuarios en Honduras?

    \item ¿Qué usuarios internos son los más claves para preguntas específicas?

    \item ¿Qué usuarios internos son los más claves para preguntas generales?

    \item ¿Qué modelo o modelos estadísticos se utilizan para generar muestreos estratificados a partir de un conjunto de datos?

    \item ¿Cuáles son los estratos de suelos agrícolas en Honduras?

    \item ¿Cómo es la entrada (el muestreo virtual) de los procesos de analítica para análisis de suelos agrícola en términos de transformaciones (filtrado, mapeo, etc.), estructura (desnormalizado, relacional, etc.) con respecto al muestreo físico?

    \item ¿Qué información geográfica manejan los usuarios sobre sus fincas, lotes o suelos agrícolas?

    \item ¿El muestreo físico histórico está relacionado con el actual o evoluciona en función de las zafras?

    \item ¿Qué tipo de cultivo/cosecha/variedad se maneja en las fincas de Honduras? ¿Qué relaciones (principalmente biológicas) hay entre ellas?
\end{itemize}
