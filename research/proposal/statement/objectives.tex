% Copyright (c) 2021 Tobias Briones. All rights reserved.
% SPDX-License-Identifier: CC-BY-4.0
%
% This file is part of https://github.com/tobiasbriones/
% cp-unah-mm700-agricultural-soil-sampling-for-data-analysis
%
% This source code is licensed under the Creative Commons Attribution 4.0
% International License found in the LICENSE-CC file in the root directory of
% this source tree or at https://spdx.org/licenses/CC-BY-4.0

\subsection{Objetivos}

\subsection{Objetivo General}

Modelar un proceso en el cual se obtengan muestras de lotes de suelo agrícola dado las siguientes condiciones:

\begin{itemize}
    \item Asumir un conjunto de datos ya muestreados físicamente (por parte de la empresa agrícola) como entrada del proceso.

    \item El muestreo es estratificado basado en el MAS.

    \item El proceso es de utilidad para las etapas iniciales de analítica.

    \item La salida del proceso (el muestreo) permite posteriormente entrenar modelos de analítica más eficientemente al minimizar sus entradas con respecto a la muestra original.
\end{itemize}

\subsection{Objetivos Específicos}

\begin{itemize}
    \item Encontrar inductivamente modelos generales para analizar los suelos y lotes.

    \item Diseñar un proceso para determinar un muestreo estratificado de lotes como producto mínimo viable para este estudio.

    \item Identificar preguntas clave para obtener estratos a partir de encuestas a los usuarios.

    \item Establecer un marco teórico estadístico para medir las variables en cuestión.

    \item Diseñar un modelo de selección de lotes por homogeneidad.

    \item Medir las relaciones para determinar cuan fina será la partición.

    \item Concluir en un resultado final que pueda ser utilizado por software de analítica.

    \item Proveer las herramientas de análisis de datos para visualizar y configurar la entrada y salida del proceso.
\end{itemize}
