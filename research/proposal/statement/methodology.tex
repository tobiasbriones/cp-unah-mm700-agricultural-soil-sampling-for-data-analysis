% Copyright (c) 2021 Tobias Briones. All rights reserved.
% SPDX-License-Identifier: CC-BY-4.0
%
% This file is part of https://github.com/tobiasbriones/
% cp-unah-mm700-agricultural-soil-sampling-for-data-analysis
%
% This source code is licensed under the Creative Commons Attribution 4.0
% International License found in the LICENSE-CC file in the root directory of
% this source tree or at https://spdx.org/licenses/CC-BY-4.0

\subsection{Metodología}

La metodología para llevar a cabo este proyecto es de carácter mixta, es decir, cuantitativa y cualitativa. En el análisis cuantitativo se tienen los modelos estadísticos y de ciencia de datos que son computados por las bibliotecas de Python. Por otra parte, los datos que se analizan en agricultura y recolectados mediante encuestas y filtración de datos crudos (iniciales) tienen etiquetas que dan valor de negocio (zafra, cosecha, rendimiento, etc.).

\bigbreak

Con respecto al enfoque cuantitativo se tiene que este estudio es de carácter deductivo al aplicar métodos estadísticos y de teoría de muestreo además de reducir el uso de la computación a dominio específico para dicho problema de investigación. Por otra parte, se deberá de generalizar el modelo a plantear a partir de la información agrícola del área local y sus necesidades. En tanto a encuestas, esto conduce a utilizar una parte del método cualitativo para, por ejemplo, obtener el tipo de cosechas que se producen en la región y que serán capturadas por el muestreo correspondiente. La parte cualitativa ha de permitir poder establecer la homogeneidad entre los estratos para el muestreo estratificado.

\subsection{Revisión de Bibliografía}

Se consultó con diversas fuentes de información para obtener un marco teórico que permitió desprender todos los detalles necesarios para el artefacto de investigación incluyendo definiciones y procesos de agricultura en organizaciones pertinentes al tema como la FAO.org, cálculos estadísticos de libros de muestreo y documentación sobre las potenciales herramientas en Python para este tipo de aplicaciones.
