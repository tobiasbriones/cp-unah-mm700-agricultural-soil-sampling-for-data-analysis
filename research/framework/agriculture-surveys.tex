% Copyright (c) 2021 Tobias Briones. All rights reserved.
% SPDX-License-Identifier: CC-BY-4.0
%
% This file is part of https://github.com/tobiasbriones/
% cp-unah-mm700-agricultural-soil-sampling-for-data-analysis
%
% This source code is licensed under the Creative Commons Attribution 4.0
% International License found in the LICENSE-CC file in the root directory of
% this source tree or at https://spdx.org/licenses/CC-BY-4.0

\subsection{Encuestas Agrícolas}

A fin de obtener información precisa para poder diseñar un muestreo representativo y eficiente puede ser necesario hacer un análisis mediante encuestas y poder obtener uno de los modelos de muestreo que se conocen.

\bigbreak

Las encuestas agrícolas son muy complicadas y tienen un gran número de variables las cuales pueden comprender en tanto de \cite{organizacion-de-las-naciones-unidas-para-la-agricultura-y-la-alimentacion-1990}:

\begin{itemize}
    \item Herramientas, maquinarias, fertilizantes, plaguicidas, semillas y otros insumos.
    \item Superficie total y régimen de tenencia.
    \item Superficie bajo cultivos y producción.
    \item Verduras, frutas, nueces.
    \item Ganadería, avicultura, estabulación.

    \item Pesca, caza y explotación maderera.
    \item Regadío, pozos, avenamiento y cercado.
    \item Ingresos, comercialización, gastos, ahorros.
    \item Recuento y características de la población; mano de obra no pagada.
    \item Sanidad, educación, ocupación y estadísticas sociales de la población agrícola.
    \item Hogares y edificios agrícolas.
    \item Transporte y comunicaciones de la población agrícola.
    \item Fuentes y consumo de alimentos.
    \item Encuestas de opinión acerca de las políticas, métodos, productos, etc.
\end{itemize}

\bigbreak

Además de utilizar estas variables, se puede obtener más datos a partir de variables auxiliares. Con respecto a las fuentes de datos se puede acceder a los productores y las operaciones agrícolas, hogares agrícolas en relación con otros datos.

\bigbreak

Las encuestas agrícolas suelen ser muy variadas, cada variable que se ha enumerado tiene sus propias variaciones y propósitos múltiples. Por esto observamos que se deberá recurrir a \textbf{encuestas multitemáticas}. Además, también se necesitan aplicar métodos múltiples ya que las mediciones que se hacen requieren métodos diferentes de medición.
