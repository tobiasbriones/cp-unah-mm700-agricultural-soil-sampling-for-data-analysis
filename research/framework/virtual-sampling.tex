% Copyright (c) 2021 Tobias Briones. All rights reserved.
% SPDX-License-Identifier: CC-BY-4.0
%
% This file is part of https://github.com/tobiasbriones/
% cp-unah-mm700-agricultural-soil-sampling-for-data-analysis
%
% This source code is licensed under the Creative Commons Attribution 4.0
% International License found in the LICENSE-CC file in the root directory of
% this source tree or at https://spdx.org/licenses/CC-BY-4.0

\subsection{Muestreo Virtual}

En esta sección se documenta las diferentes salidas computacionales que son de gran importancia para trabajar con muestras de datos. Actualmente, el lenguaje de programación adoptado por la comunidad de científicos de datos y otras comunidades similares es Python debido a la estandarización que tienen sus bibliotecas de matemáticas y ciencia de datos, además de la simplicidad de su sintaxis lo cual permite la creación de prototipos y de documentos interactivos con herramientas como Jupyter Notebook \footnote{Jupyter Notebook es una aplicación web de código abierto que permite crear y compartir documentos que contienen código en tiempo real, ecuaciones, visualizaciones y texto narrativo. Los usos incluyen: limpieza y transformación de datos, simulación numérica, modelado estadístico, visualización de datos, aprendizaje automático y mucho más \cite{jupyter-2021}.}. Toda la gama de bibliotecas y herramientas para matemática y análisis de datos es dada en su mayoría bajo licencias de código abierto permisivas (comúnmente BSD-3-Clause). La ventaja de Python es que como ya se mencionó, ha sido altamente adoptado por la comunidad por su facilidad de entrada para no-programadores y también a destacar que es un lenguaje de scripting \footnote{Un lenguaje de scripting es un lenguaje de programación para un entorno de ejecución (runtime) que automatiza la ejecución de tareas que, de otro modo, serían realizadas individualmente por un operador humano \cite{wikipedia-scripting-2021B}.} por lo que su función principal es servir de interfaz (wrapper) para lenguajes de programación que suelen ser robustos, eficientes y \textit{sin abstracciones innecesarias} como C y Rust \footnote{La tendencia actual para modelos reales (no-prototipos) está empezando a emplear Rust en lugar de lenguajes similares como C y C++ y de lenguajes de entrada como Python, mostrado también por la revista Nature en \say{Why scientists are turning to Rust} \cite{nature-editorial-2020}.}.

\subsubsection{Enlaces}

Se enlistan las fuentes de los proyectos de relevancia que se necesitan para esta sección:

\begin{itemize}
    \item Python: \url{https://www.python.org}.
    \item Pandas: \url{https://pandas.pydata.org}.
    \item Matplotlib: \url{https://matplotlib.org}.
    \item GeoPandas: \url{https://geopandas.org}.
    \item NumPy: \url{https://numpy.org}.
    \item Plotly: \url{https://plotly.com}.
    \item Jupyter: \url{https://jupyter.org}.
    \item SciPy: \url{https://scipy.org}.
    \item StatsModels: \url{https://www.statsmodels.org}.
\end{itemize}

\subsubsection{Visualización de Datos}

La visualización de datos es fundamental en la ciencia de datos de forma que se puede tener mucha información de forma compacta y estudiar los resultados al mismo tiempo.

\bigbreak

Los dos usos principales de la visualización de datos son \cite{grus-2015}:

\begin{itemize}
    \item Explorar datos.
    \item Comunicar datos.
\end{itemize}

La biblioteca matplotlib es sugerida \cite{grus-2015} para la visualización de datos simples que no requieran un entorno web. Las posibilidades son: gráficos de barra simples, gráficos de líneas y gráficos de dispersión.

\bigbreak

Otras bibliotecas de interés comprenden \textbf{seaborn} derivado de matplotlib para visualizaciones más complejas y bonitas, \textbf{D3.js} para visualizaciones sostificadas e interactivas para la web, \textbf{Bokeh} para visualizaciones $3D$ en Python y \textbf{ggplot} el cual es una versión portada de la biblioteca ggplot2 del lenguaje R.

\subsubsection{Estadística}

Las bibliotecas de SciPy, pandas y StatsModels tienen una gran gama de funciones para aplicaciones de estadística. Por el resto, Python trae muchas construcciones para el manejo de funciones, vectores, operaciones y matrices aunque las matrices no son lo más fuerte de Python.

\subsubsection{GIS en la Agricultura}

Con respecto a la visualización de los datos, se tiene en consideración que el sector agrícola cuenta o debería contar con el uso de GISs. Un \textbf{GIS (Geographic Information System)} es una herramienta que crea representaciones visuales de los datos y realiza análisis espaciales a fin de tomar decisiones informadas; combina hardware, software y datos \cite{hammonds-2019}.

\bigbreak

En agricultura, un GIS puede dar información sobre el suelo, en tanto a cómo está estructurado, las propiedades y condiciones que presenta. Esto también es muy útil para visualizar un muestreo de puntos en un mapa en $2D$. Los archivos con los que se suele trabajar estos gráficos son archivos shapefile.

\bigbreak

\textbf{Shapefile:} \say{Formato de datos utilizado por la mayoría de software GIS/FMIS para almacenar datos espaciales (por ejemplo, límites de campo, ubicaciones de puntos o los polígonos que representan celdas de la cuadrícula). Un shapefile se almacena como una colección de archivos con un nombre común pero con diferentes extensiones de archivo (por ejemplo, .shp, .shx y .dbf).} Fuente: NC State Extension \cite{nc-state-extension-2021}.

\bigbreak

La biblioteca GeoPandas es una buena opción para manipular archivos shapefile y otros gráficos vectoriales.
