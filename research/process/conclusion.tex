% Copyright (c) 2021 Tobias Briones. All rights reserved.
% SPDX-License-Identifier: CC-BY-4.0
%
% This file is part of https://github.com/tobiasbriones/
% cp-unah-mm700-agricultural-soil-sampling-for-data-analysis
%
% This source code is licensed under the Creative Commons Attribution 4.0
% International License found in the LICENSE-CC file in the root directory of
% this source tree or at https://spdx.org/licenses/CC-BY-4.0

\section{Conclusión}

Se desarrolló un modelo de muestro virtual para suelo agrícola que consiste en una herramienta mínima viable como artefacto de investigación. El modelo implementó una API que es de utilidad para el analista de datos y puede ser configurada y extendida en funcionalidad.

\bigbreak

Se desarrolló un marco teórico para entender mejor ambas partes interesadas, a saber, la empresa de analítica y agrícola con cierto enfoque en Honduras. Esto sirve al analista a poder configurar el modelo de muestreo y obtener información clave sobre como definir los estratos y la obtención del SIG del suelo apropiada para la visualización del muestreo. Se ha tomado en cuenta el uso de encuestas como herramienta de análisis de sistemas para lograr este objetivo.

\bigbreak

Finalmente, se demostró el funcionamiento y eficiencia del prototipo con datos irreales y sin mucho sentido debido a las complicaciones para obtener un conjunto de datos real de acceso libre. Esto indica que el analista puede generalizar el proceso de muestreo en una base de caso por caso y que al aplicar la herramienta de muestreo será de utilidad para un caso de uso real.

\subsection{Recomendaciones}

Una vez aplicados los conocimientos y resultados de la tesis se recomienda:

\begin{itemize}
    \item Proyecto en \href{https://github.com/tobiasbriones/cp-unah-mm700-agricultural-soil-sampling-for-data-analysis}{GitHub}.

    \item Dar uso a la POO para crear APIs que sean útiles y reutilizables.

    \item Extender el modelo con cálculos estadísticos para reportar y automatizar la selección de valores (e.g. $h_n$).

    \item Crear más documentación de alto nivel para entender el proceso de muestreo al trabajar en equipo y crear aplicaciones.
\end{itemize}
